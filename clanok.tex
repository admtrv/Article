\documentclass[10pt,oneside,english,a4paper]{article}
\usepackage[english]{babel}
\usepackage[IL2]{fontenc}
\usepackage[utf8]{inputenc}
\usepackage{graphicx}
\usepackage{url} 
\usepackage{hyperref} 
\usepackage{cite}
\usepackage{listings}
\usepackage[margin=3cm]{geometry}

\pagestyle{headings}

\title{How online maps search works: technologies, algorithms and inner structure.
\thanks{Semester project in the subject Methods of engineering work, academic year 2023/24, leadership: PaedDr. Pavol Batalik}}

\author{Anton Dmitriev\\[2pt]
	{\small Slovak University of Technology in Bratislava}\\
	{\small Faculty of Informatics and Information Technologies}\\
	{\small \texttt{xdmitriev@stuba.sk}}
	}

\date{\small \today} 



\begin{document}

\maketitle

\begin{abstract}
	Modern online maps have become an integral part of our daily life, providing us with quick access to geographical information and real-time location. However, behind this wonderful opportunity lies a technically complicated system that has been developing and improving for many years. This article provides an overview of the historical and technological development of online mapping and aims to provide an understanding of how modern online maps function and how they provide users with the data they need. Readers will learn how the evolution of technologies in map data has led to the ability to quickly and accurately find places and addresses on a map. Because, in an era where mobile devices and apps are becoming an essential part of our lives, it is important to understand how these maps collect and store information about the world around us, and how they help us navigate and find the places we need.
\end{abstract}





\bibliography{literatura}
\bibliographystyle{plain}
\end{document}
